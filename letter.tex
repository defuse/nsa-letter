\documentclass{letter}
\usepackage[a4paper]{geometry}
\signature{TODO: YOUR NAME HERE}
\address{TODO: YOUR NAME HERE \\ YOUR STREET ADDRESS \\ YOUR CITY \ YOUR POSTAL CODE}
\begin{document}
\begin{letter}{TODO: YOUR MP \\ House of Commons\\ Ottawa, ON \ K1A 0A6}
\opening{Dear TODO: YOUR MP:}

It has been revealed that the National Security Agency (NSA), an intelligence
agency of the United States Department of Defense, has been collecting telephone
call data on millions of innocent Verizon customers [1], and has ordered large
Internet companies including Google, Facebook, and Apple to give the NSA access
to their computer systems for the purpose of spying on their users [2].

The actions of the NSA are in clear violation of the Fourth Amendment to the
United States Constitution, which reads as follows.

\begin{quote}
\emph{
The right of the people to be secure in their persons, houses, papers, and
effects, against unreasonable searches and seizures, shall not be violated, and
no warrants shall issue, but upon probable cause, supported by oath or
affirmation, and particularly describing the place to be searched, and the
persons or things to be seized. 
}
\end{quote}

The Canadian Government must take every precaution to ensure that such an abuse
of power is never carried out by the Canadian Government or an affiliated
intelligence agency. Governmental transparency must be maintained.

The NSA's spying does not only affect Americans, but Canadians, too. The order
forcing Verizon to provide the NSA with telephone call data [3] includes calls
``between the United States and abroad.'' If Canada is to continue to be
a leader in human rights issues, the Canadian Government must put pressure on
the United States to remedy these blatant abuses of power.

\begin{verbatim}
[1] http://bit.ly/191pQJY
[2] http://bit.ly/1baaUGj
[3] http://bit.ly/18QRuXx
\end{verbatim}

\closing{Sincerely,}
\end{letter}
\end{document}
